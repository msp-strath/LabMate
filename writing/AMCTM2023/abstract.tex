\documentclass{ws-procs9x6}

%%%% Packages %%%%%%%%%%%%%%%%%%%%%%%%%%%%%%%%%%%%%%%%%%%%
\usepackage[colorlinks=true]{hyperref}
\usepackage{microtype}
%%%%%%%%%%%%%%%%%%%%%%%%%%%%%%%%%%%%%%%%%%%%%%%%%%%%%%%%%%

%%%% Macros %%%%%%%%%%%%%%%%%%%%%%%%%%%%%%%%%%%%%%%%%%%%%%
\newcommand{\pct}{\texttt{\symbol{37}}}
\newcommand{\dir}{\texttt{\symbol{62}}}
\newcommand{\res}{\texttt{\symbol{60}}}
\newcommand{\lcb}{\texttt{\symbol{123}}}
\newcommand{\rcb}{\texttt{\symbol{125}}}
\newcommand{\lsb}{\texttt{\symbol{91}}}
\newcommand{\rsb}{\texttt{\symbol{93}}}

\newcommand{\lr}{\textsc{LabMate}}
\newcommand{\repo}{\url{https://github.com/msp-strath/LabMate}}
\newcommand{\ma}{\textsc{MATLAB}}
\newcommand{\remph}{\emph}
%%%%%%%%%%%%%%%%%%%%%%%%%%%%%%%%%%%%%%%%%%%%%%%%%%%%%%%%%%

\begin{document}

\title{\lr --- an interactive assistant for \ma}
\author{Conor McBride, Georgi Nakov, and Fredrik Nordvall Forsberg}
\address{University of Strathclyde}

\bodymatter

\paragraph{Introduction}

We set out the prospectus for and report our progress on \lr, an interactive computer system designed to help human users of \ma{} write programs which actively make more of the sense those humans intend. This work is supported by the UK's National Physical Laboratory, and is significantly informed by a code corpus arising from their work, along with observations of their praxis. The latter is solid from an engineering perspective, with a discipline of documenting the structure and significance of each variable in \remph{comments} which, although not formal, exhibit a certain regularity of format. That is to say, the humans communicate to \remph{one another} information vital for comprehension of their software, whilst withholding the very same from the \remph{machine}. We can and should do better.

%\paragraph{Ulterior motive}

Our goal is to enhance practitioners' experience of their existing toolchains, rather than to insist on the adoption of a radically different language prototype. \lr{} gives a formal language of comments for \ma{} programs, and can thus help both preventatively (with safety checks ahead of execution) and constructively (generating correct code from richer information). Our approach is inspired by Andrew Kennedy's work at Facebook on adding types incrementally to PHP~\cite{hack}.

\paragraph{Directives and Responses}

\lr{} introduces a language of \emph{directives}:
\[\pct\dir\;\mathit{directive}
\]
which \ma{} treats as comments, but \lr{} acts on, transforming code with directives to a new version of that code in four ways:
\begin{itemize}
\item documenting \remph{types} of functions and variables, giving not only data representation, but also richer metadata, e.g., units of measure;
\item querying the type of a part of the program, or a gap in it;
\item requesting the systematic transformation of the user's code, e.g., consistently renaming all occurrences of a given variable;
\item generating low-level code serving some stated high-level purpose on the basis of supplied metadata, e.g., to read a data set from a file.
\end{itemize}
Programmers can run \lr{} iteratively on their work in progress, gaining support, feedback, and the alleviation of systematic drudgery. Controlled only by the directives in its input, \lr\ may readily be integrated with existing editors with minimal assumptions as to which. It communicates back by including \emph{responses} (comments with useful information)
\[
\pct\res\;\mathit{response}
\]
and \emph{syntheses} (code segments unlikely to be of interest)
\[\begin{array}{l}
\pct\res\lcb\qquad \mbox{(which is a \remph{single line} \ma{} comment)}\\
\mathit{synthesized}\; \ma\;\mathit{code}\\
\pct\res\rcb\qquad \mbox{(also a \remph{single line} \ma{} comment)} \\
\end{array}\]

\paragraph{\lr{} types}

The type of a \ma{} matrix is given by, at worst, its sizes in each dimension and cell type. However, a size is but a number and thus a special case of a list of \remph{headers}, such as you might find in a spreadsheet: the type of each cell may depend on the headers which govern its position. A key use of this refinement takes headers to be \remph{units}: \lr{} will check that the units of each cell respect these headers so that inner products computed during matrix multiplication are dimensionally meaningful~\cite{dimTypes}.

\paragraph{Current status}

\lr{} is in development and available at \repo. We can currently parse most \ma{} code we have encountered in the wild, and have implemented directives for renaming variables and generating code for reading data sets from files based on \textsc{MGen} input descriptions~\cite{mgen}.

\bibliographystyle{ws-procs9x6}
\bibliography{labrat}


\end{document}
