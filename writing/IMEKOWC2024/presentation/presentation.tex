\documentclass[]{beamer}
\usepackage[utf8x]{inputenc}
\usepackage{pgfpages}
\usepackage{hyperref}
\usepackage{listings}
\usepackage{lstautogobble}
\usepackage{xspace}
\usepackage{xparse}
\usepackage{matlab-prettifier}
%\setbeameroption{show notes on second screen}

\usepackage{microtype}

\usepackage{tikz}
\usetikzlibrary{shapes.callouts, shapes.geometric, arrows, arrows.meta, chains}

% speech bubbles
\tikzset{
  invisible/.style={opacity=0,text opacity=0},
  visible on/.style={alt=#1{}{invisible}},
  alt/.code args={<#1>#2#3}{%
    \alt<#1>{\pgfkeysalso{#2}}{\pgfkeysalso{#3}} % \pgfkeysalso doesn't change the path
  },
}
\newcommand{\tikzmark}[1]{\tikz[overlay,remember picture,baseline=-0.5ex] \node (#1) {};}

\NewDocumentCommand{\mycallout}{r<> O{opacity=0.8,text opacity=1} m m}{%
  \tikz[remember picture, overlay]\node[align=center, fill=blue!20, text width=12em,
  #2,visible on=<#1>, rounded corners,
  draw,rectangle callout,anchor=pointer,callout relative pointer={(180:1em)}]
  at (#3) {#4};
}
\NewDocumentCommand{\mycalloutup}{r<> O{opacity=0.8,text opacity=1} m m}{%
  \tikz[remember picture, overlay]\node[align=center, fill=blue!20, text width=12em,
  #2,visible on=<#1>, rounded corners,
  draw,rectangle callout,anchor=pointer,callout relative pointer={(240:1em)}]
  at (#3) {#4};
}


% flowcharts
\tikzset{
  reg/.style={
    rectangle,
    rounded corners,
    minimum width=3cm,
    minimum height=1cm,
    align=center,
    draw=black
  },
  arrow/.style={thick,->},
  fade/.style={
    rectangle,
    rounded corners,
    minimum width=3cm,
    minimum height=1cm,
    align=center,
    draw=gray!30,
    text=gray!30,
  },
  fadearrow/.style={thick,->, gray!30},
  highlight/.style={
    rectangle,
    rounded corners,
    minimum width=3cm,
    minimum height=1cm,
    align=center,
    draw=black,
    fill=blue!10
  },
  end/.style={
    tape,
    tape bend top=none,tape bend height=2mm,
    tape bend bottom=out and in,
    minimum width=3cm,
    minimum height=1cm,
    align=center,
    draw=black
  },
  fadeend/.style={
  tape,
  tape bend top=none,tape bend height=2mm,
  tape bend bottom=out and in,
  minimum width=3cm,
  minimum height=1cm,
  align=center,
  draw=gray!30,
  text=gray!30
  }
}

\lstset{ %
  language=Matlab,
  style=Matlab-editor,
  autogobble=true,
  escapechar=!,
  belowskip=0em
}

\newcommand{\pct}{\texttt{\symbol{37}}}
\newcommand{\dir}{\texttt{\symbol{62}}}
\newcommand{\res}{\texttt{\symbol{60}}}
\newcommand{\lcb}{\texttt{\symbol{123}}}
\newcommand{\rcb}{\texttt{\symbol{125}}}
\newcommand{\lsb}{\texttt{\symbol{91}}}
\newcommand{\rsb}{\texttt{\symbol{93}}}
\newcommand{\keyword}[1]{\textcolor{blue}{#1}}
\newcommand{\lm}{\textsc{LabMate}\xspace}
\newcommand{\repo}{\url{https://github.com/msp-strath/LabMate}}
\newcommand{\ma}{\textsc{Matlab}\xspace}
\newcommand{\List}{\mathsf{List}}
\newcommand{\Unit}{\mathsf{Unit}}
\newcommand{\Abel}{\mathsf{Abel}}
\newcommand{\Int}{\mathsf{Int}}
\newcommand{\Matrix}{\mathsf{Matrix}}
\newcommand{\Enum}{\mathsf{Enum}}
\newcommand{\LL}{\mathsf{L}}
\newcommand{\TT}{\mathsf{T}}
\newcommand{\MM}{\mathsf{M}}
\newcommand{\Kg}{\mathsf{kg}}
\newcommand{\Sec}{\mathsf{s}}
\newcommand{\Km}{\mathsf{m}}
\newcommand{\NameTimes}{\!\!\times\!\!}

\setbeamertemplate{frametitle}[default][center]
\beamertemplatenavigationsymbolsempty

\defbeamertemplate*{footline}{my theme}
{
  \leavevmode%
  \hbox{%
    \usebeamerfont{date in head/foot}
    \hspace*{0.97\paperwidth}\raisebox{2pt}{\insertframenumber{}}}
  \vskip0pt%
}


\title{\huge \lm: \\ \centerline{Supporting Types for \ma} }
\author[McBride, Nakov, Nordvall Forsberg et al]{\small Conor McBride$^{1}$, \underline{Georgi Nakov}$^{1}$, Fredrik Nordvall Forsberg$^{1}$,\\ Andr\'e Videla$^{1}$, Alistair Forbes$^{2}$, Keith Lines$^{2}$}
\institute[]{$^{1}$University of Strathclyde, UK\\$^{2}$National Physical Laboratory, UK}

\begin{document}

\begin{frame}[plain]
  \titlepage
\end{frame}

\setcounter{framenumber}{0}

\begin{frame}{The Problem}
  \begin{itemize}
  \item Much software in science and engineering.
    written in \ma
    \smallskip
    \begin{itemize}
    \item May contain errors and bugs, as with any software.
    \end{itemize}
    \bigskip
  \item Developers often leave comments about how their data should be
    interpreted, e.g., units of measure for quantities.
%  \item Developers often leave comments what are the corresponding
%    physical systems of their data and how it should be interpreted,
%    e.g., units of measure for quantities.
    \bigskip
  \item However \ma is oblivious to these high-level comments, and
    instead performs low-level %compatibility
    checks during execution.
  \end{itemize}
\end{frame}

\begin{frame}{Our Plan}
  \textit{Can we do better?}
  
  \begin{itemize}
    \medskip
  \item Make these developer comments formal.
    \medskip
  \item \ldots and create a tool to make use of them --- \lm.
    \begin{itemize}
      \smallskip
    \item Keep existing \ma code and toolchains; no need to
      switch to a new language.
    \end{itemize}
    \medskip
  \item Distill the essence of the developer comments in
    \lm's expressive type system.
    \begin{itemize}
      \smallskip
    \item A set of logical rules that assign domains of admissible
      values to program expressions.
    \end{itemize}
    \medskip
  \item \lm is meant to be used while writing the code to get instant
    feedback and guidance --- do not delay until execution.
  \end{itemize}
\end{frame}

\definecolor{commentgreen}{RGB}{034,139,034}

\begin{frame}[fragile]{How does \lm Work?}
  \begin{itemize}
  \item \lm is a program transducer: reads \ma code with formal comments, and outputs a modified version of the input.
    \medskip
  \item These formal comments are directives --- they start with \textcolor{commentgreen}{\texttt{\%<}}.
    \medskip
  \item Input the program:
   \begin{lstlisting}[xleftmargin=2em]
    %> rename n x !\tikzmark{ren}!
    n = 5;
    display(n);
   \end{lstlisting}
  \pause
  \mycallout<+>{ren}{this is an input directive to rename a variable}
  \pause
\item Run \lm to get:
  \begin{lstlisting}[xleftmargin=2em]
%< LabMate 0.2.0.1 !\tikzmark{ver}!
%< renamed n x !\tikzmark{renr}!
x = 5;
display(x);
   \end{lstlisting}
  \pause
  \mycallout<+>{renr}{LabMate response to the input directive}
  \mycallout<+>{ver}{mark the file as processed by LabMate}
\end{itemize}
\end{frame}

\begin{frame}[fragile]{Matrix Types}
  \begin{itemize}[<+->]
  \item Matrices feature heavily in \ma code.
    \medskip
  \item \lm supports type annotations for matrices:
\begin{lstlisting}[xleftmargin=0em, belowskip=-0.5em]
%> A :: [ 1 x 2 ] int  !\tikzmark{ann1}!
A = [ 3 4 ]
%> B :: [ 2 x 4 ] int  !\tikzmark{ann2}!
B = [ 1 1 1 1
      5 6 7 8 ]
C = A * B
\end{lstlisting}
  \pause
  \mycallout<+>{ann1}{type annotations at declaration of A}
  \mycallout<+>{ann2}{type annotations at declaration of B}
  \item We can ask for type information:
\begin{lstlisting}[xleftmargin=0em, belowskip=-0.5em]
%> typeof C !\tikzmark{query}!
%< C :: [Matrix 1 4 int]!\tikzmark{resp}!
\end{lstlisting}
 \pause
  \mycallout<+>{query}{query for the type of C}
  \mycalloutup<+>{resp}{LabMate can infer the dimensions}
  \item \ldots and can easily spot incompatible sizing:
\begin{lstlisting}[xleftmargin=0em]
D = B * A
%> typeof D
%< The expression D is!\tikzmark{bad}! quite a puzzle
\end{lstlisting}
    \pause
      \mycalloutup<+>{bad}{LabMate can point out an error with D}
\end{itemize}
\end{frame}

\begin{frame}[fragile]{Matrix Types}
  \begin{itemize}[<+->]
  \item \lm processes arbitrary \ma code.
    \smallskip
    \begin{itemize}
    \item \lm does not rely on constants, variables work as well.
    \end{itemize}
  \item
\begin{lstlisting}[xleftmargin=0em]
function B = f(A)
  %> B :: [ 1 x 3 ] int
  B = [ 1 A ]

  %> typeof A
  %< A :: [Matrix int 1 2]!\tikzmark{fromb}!
end

A = 'hello'
%> typeof A
%< A :: string!\tikzmark{scope}!
\end{lstlisting}
   \pause
   \mycalloutup<+>{fromb}{LabMate infers type of A from the annotation on B}
   \mycalloutup<+>{scope}{tracks Matlab scope}
  \end{itemize}
\end{frame}

\begin{frame}[fragile]{Dimensions and Quantities}
  \begin{itemize}[<+->]
  \item \lm has support for arbitrary quantities.
    \begin{lstlisting}[xleftmargin=0em,belowskip=-1em]
%> dimensions V for Q over `Mass!\tikzmark{base}!, `Time
%> unit kg :: Q({!\tikzmark{unit1}! `Mass })!\tikzmark{unit}!
\end{lstlisting}
%\pause
   \mycalloutup<+>{base}{define some base set of dimensions}
   \mycalloutup<+>{unit}{and a canonical unit of measure}
   \mycalloutup<+>{unit1}{can be arbitrary group expression over V}
\pause
\begin{lstlisting}[xleftmargin=0em]
%<{
kg = 1; !\tikzmark{kg}!
%<}
\end{lstlisting}
\pause
   \mycallout<+>{kg}{this is a ``magic'' response that LabMate emits}
\pause
\item We can then use the unit of measure for quantities:
\begin{lstlisting}[xleftmargin=0em]
y = 5*kg !\tikzmark{q}!
%> typeof y
%< y :: Quantity (Enum [`Mass, `Time])
%<               {`Mass}
\end{lstlisting}
%  \pause
   \mycallout<+>{q}{turn a value of a dimensionless type into a quantity}
  \end{itemize}
\end{frame}

\begin{frame}[fragile]{Dimensional Consistency for Matrices}
  \begin{itemize}[<+->]
  \item Most \ma programs do not work on uniform matrices: type of the entry $e_{i, j}$ might depend on the indices $i$ and $j$.
    \medskip
  \item A common scenario when working with matrices of quantities
%    $\begin{array}{ccccc}
% e_{1,1}  &        & e_{1,j} &        & e_{1,n} \\
%         &        &         &        &         \\
% e_{i,1}  &        & e_{i,j}\tikzmark{mat} &        & e_{i,n} \\
%         &        &         &        &         \\
% e_{m,1}  &        & e_{m,j} &        & e_{m,n}  \\
%\end{array}$
%    \mycallout<+>{mat}{the type might depend on i and j}
\begin{lstlisting}[xleftmargin=0em]
%> dimensions V for Q over `L, `M, `T
%> unit metre :: Q({ `L })
%> unit kg :: Q({ `M })
%> unit sec :: Q({ `T })
\end{lstlisting}
\medskip
\item Work in progress: \lm support for such matrices
\begin{lstlisting}[xleftmargin=0em]
% > A :: [ i <- [{} {`T}]
%        x j <- [{} {`L}]
%        ] Q({`M * j / i})
A = [ 2*kg       5*kg*metre
      3*kg/sec   4*kg*metre/sec ]
\end{lstlisting}
  \end{itemize}
\end{frame}

\begin{frame}{Implementation Details}
  \begin{itemize}[<+->]
  \item \ma programs are modelled as trees of commands, rather than sequence of commands.
    \medskip
    \begin{itemize}
    \item Type information is propagated (consistently) throughout the tree; can put type annotations after variable declarations.
      \medskip
    \item Not every annotation is needed; document the interesting ones.
    \end{itemize}
    \medskip
  \item \ma expressions are translated to \lm internal core type theory.
    \medskip
    \begin{itemize}
    \item Matrix types are parametrised over 5 parameters with dependencies between them.
      \medskip
    \item Quantities are modelled as the free Abelian group over a base set of dimensions.
      \medskip
    \item The typechecker understands nontrivial algebraic properties.
    \end{itemize}
  \end{itemize}
\end{frame}

\begin{frame}{Current Progress \& Future Plans}
  \begin{itemize}[<+->]
  \item \lm is under active development.
    \smallskip
    \begin{itemize}
    \item Available on GitHub, please get it in touch if interested.
    \end{itemize}
    \bigskip
  \item Work in the pipeline:
    \smallskip
    \begin{itemize}
    \item \keyword{Uniqueness of representation}: currently, a matrix with quantities can have more than one corresponding type; this might lead to odd behaviour during typechecking.
      \medskip
    \item \keyword{Quality of life improvements}: better messages and more readable responses from \lm.
      \medskip
      \end{itemize}
  \item We want to extend our coverage to loops and conditionals in the future.
  \end{itemize}
  %\pause
  %\large \centerline{Thank you for your attention!}
\end{frame}



\end{document}
%%% Local Variables:
%%% mode: latex
%%% TeX-master: t
%%% End:
