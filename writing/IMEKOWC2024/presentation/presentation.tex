\documentclass[]{beamer}
\usepackage[utf8x]{inputenc}
\usepackage{pgfpages}
\usepackage{hyperref}
\usepackage{listings}
\usepackage{lstautogobble}
\usepackage{xspace}
\usepackage{xparse}
\usepackage{matlab-prettifier}
%\setbeameroption{show notes on second screen}

\usepackage{microtype}

\usepackage{tikz}
\usetikzlibrary{shapes.callouts, shapes.geometric, arrows, arrows.meta, chains}

% speech bubbles
\tikzset{
  invisible/.style={opacity=0,text opacity=0},
  visible on/.style={alt=#1{}{invisible}},
  alt/.code args={<#1>#2#3}{%
    \alt<#1>{\pgfkeysalso{#2}}{\pgfkeysalso{#3}} % \pgfkeysalso doesn't change the path
  },
}
\newcommand{\tikzmark}[1]{\tikz[overlay,remember picture,baseline=-0.5ex] \node (#1) {};}

\NewDocumentCommand{\mycallout}{r<> O{opacity=0.8,text opacity=1} m m}{%
  \tikz[remember picture, overlay]\node[align=center, fill=blue!20, text width=12em,
  #2,visible on=<#1>, rounded corners,
  draw,rectangle callout,anchor=pointer,callout relative pointer={(180:1em)}]
  at (#3) {#4};
}
\NewDocumentCommand{\mycalloutup}{r<> O{opacity=0.8,text opacity=1} m m}{%
  \tikz[remember picture, overlay]\node[align=center, fill=blue!20, text width=12em,
  #2,visible on=<#1>, rounded corners,
  draw,rectangle callout,anchor=pointer,callout relative pointer={(240:1em)}]
  at (#3) {#4};
}


% flowcharts
\tikzset{
  reg/.style={
    rectangle,
    rounded corners,
    minimum width=3cm,
    minimum height=1cm,
    align=center,
    draw=black
  },
  arrow/.style={thick,->},
  fade/.style={
    rectangle,
    rounded corners,
    minimum width=3cm,
    minimum height=1cm,
    align=center,
    draw=gray!30,
    text=gray!30,
  },
  fadearrow/.style={thick,->, gray!30},
  highlight/.style={
    rectangle,
    rounded corners,
    minimum width=3cm,
    minimum height=1cm,
    align=center,
    draw=black,
    fill=blue!10
  },
  end/.style={
    tape,
    tape bend top=none,tape bend height=2mm,
    tape bend bottom=out and in,
    minimum width=3cm,
    minimum height=1cm,
    align=center,
    draw=black
  },
  fadeend/.style={
  tape,
  tape bend top=none,tape bend height=2mm,
  tape bend bottom=out and in,
  minimum width=3cm,
  minimum height=1cm,
  align=center,
  draw=gray!30,
  text=gray!30
  }
}

\lstset{ %
  language=Matlab,
  style=Matlab-editor,
  autogobble=true,
  escapechar=!,
  belowskip=0em
}

\newcommand{\pct}{\texttt{\symbol{37}}}
\newcommand{\dir}{\texttt{\symbol{62}}}
\newcommand{\res}{\texttt{\symbol{60}}}
\newcommand{\lcb}{\texttt{\symbol{123}}}
\newcommand{\rcb}{\texttt{\symbol{125}}}
\newcommand{\lsb}{\texttt{\symbol{91}}}
\newcommand{\rsb}{\texttt{\symbol{93}}}
\newcommand{\keyword}[1]{\textcolor{blue}{#1}}
\newcommand{\lm}{\textsc{LabMate}\xspace}
\newcommand{\repo}{\url{https://github.com/msp-strath/LabMate}}
\newcommand{\ma}{\textsc{Matlab}\xspace}
\newcommand{\List}{\mathsf{List}}
\newcommand{\Unit}{\mathsf{Unit}}
\newcommand{\Abel}{\mathsf{Abel}}
\newcommand{\Int}{\mathsf{Int}}
\newcommand{\Matrix}{\mathsf{Matrix}}
\newcommand{\Enum}{\mathsf{Enum}}
\newcommand{\LL}{\mathsf{L}}
\newcommand{\TT}{\mathsf{T}}
\newcommand{\MM}{\mathsf{M}}
\newcommand{\Kg}{\mathsf{kg}}
\newcommand{\Sec}{\mathsf{s}}
\newcommand{\Km}{\mathsf{m}}
\newcommand{\NameTimes}{\!\!\times\!\!}

\setbeamertemplate{frametitle}[default][center]
\title{\huge \lm: \\ \centerline{Supporting Types for \ma} }
\author[McBride, Nakov, Nordvall Forsberg et al]{\small Conor McBride$^{1}$, \underline{Georgi Nakov}$^{1}$, Fredrik Nordvall Forsberg$^{1}$,\\ Andr\'e Videla$^{1}$, Alistair Forbes$^{2}$, Keith Lines$^{2}$}
\institute[]{$^{1}$University of Strathclyde, UK\\$^{2}$National Physical Laboratory, UK}

\begin{document}

\begin{frame}
  \titlepage
\end{frame}

\begin{frame}{The Problem}
  \begin{itemize}
  \item Loads of computational software in science and engineering is
    written in \ma
    \begin{itemize}
    \item may contain errors and bugs, as with any software
    \end{itemize}
  \item Developers often leave comments what are the corresponding
    physical systems of their data and how it should be interpreted,
    e.g., units of measure for quantities.
  \item \ma is oblivious to these high-level, semantic comments, and
    instead performs low-level compatibility checks during execution.
  \end{itemize}
\end{frame}

\begin{frame}{Our Plan}
  Can we do better?
  \begin{itemize}
  \item Make these developers' comments formal
  \item \ldots and create a tool to make use of them --- \lm
    \begin{itemize}
    \item keep the existing \ma code and toolchains, no need to
      rewrite in a new language
    \end{itemize}
  \item Distill the essence of the developers' comments in
    \lm's expressive type system
    \begin{itemize}
    \item a set of logical rules that assign a domain of admissible
      values to the expressions in our program
    \end{itemize}
  \item Run \lm multiple times while writing the code to get instant
    feedback and guidance, do not delay until execution
  \end{itemize}
\end{frame}

\begin{frame}[fragile]{How does \lm Work?}
  \begin{itemize}
  \item \lm is a program transducer: reads \ma code with formal comments, and outputs a modified version of the input
  \item These formal comments are directives --- they start with \texttt{\%<}
  \item Input the program:
   \begin{lstlisting}[xleftmargin=2em]
    %> rename n x !\tikzmark{ren}!
    n = 5;
    display(n);
   \end{lstlisting}
  \pause
  \mycallout<+>{ren}{this is an input directive to rename a variable}
  \pause
\item Run \lm to get:
  \begin{lstlisting}[xleftmargin=2em]
%< LabMate 0.2.0.0 !\tikzmark{ver}!
%< renamed n x !\tikzmark{renr}!
x = 5;
display(x);
   \end{lstlisting}
  \pause
  \mycallout<+>{renr}{LabMate response to the input directive}
  \mycallout<+>{ver}{mark the file as processed by LabMate}
\end{itemize}
\end{frame}

\begin{frame}[fragile]{Matrix Types}
  \begin{itemize}[<+->]
  \item Matrices feature heavily in \ma code
  \item \lm support type annotations for matrices
\begin{lstlisting}[xleftmargin=0em, belowskip=-0.5em]
%> A :: [ 1 x 2 ] int  !\tikzmark{ann1}!
A = [ 2 3 ]
%> B :: [ 2 x 4 ] int  !\tikzmark{ann2}!
B = [ 1 1 1 1
      3 4 5 6 ]
C = A * B
\end{lstlisting}
  \pause
  \mycallout<+>{ann1}{type annotations at declaration of A}
  \mycallout<+>{ann2}{type annotations at declaration of B}
  \item We can ask for type information
\begin{lstlisting}[xleftmargin=0em, belowskip=-0.5em]
%> typeof C !\tikzmark{query}!
%< C :: [Matrix 1 4 int]!\tikzmark{resp}!
\end{lstlisting}
 \pause
  \mycallout<+>{query}{query for the type of C}
  \mycalloutup<+>{resp}{LabMate can infer the dimensions}
  \item \ldots and can easily spot incompatible sizing
\begin{lstlisting}[xleftmargin=0em]
D = B * A
%> typeof D
%< The expression D is!\tikzmark{bad}! quite a puzzle
\end{lstlisting}
    \pause
      \mycalloutup<+>{bad}{LabMate can point out an error with D}
\end{itemize}
\end{frame}

\begin{frame}[fragile]{Matrix Types}
  \begin{itemize}[<+->]
  \item \lm processes arbitrary \ma code
    \begin{itemize}
    \item \lm does not rely on constant values, works on variables as well
    \end{itemize}
  \item
\begin{lstlisting}[xleftmargin=0em]
function B = f(A)
  %> B :: [ 1 x 3 ] int
  B = [ 1 A ]

  %> typeof A
  %< A :: [Matrix int 1 2]!\tikzmark{fromb}!
end

%> typeof A
%< The expression A is!\tikzmark{scope}! quite a puzzle
\end{lstlisting}
   \pause
   \mycalloutup<+>{fromb}{LabMate infers A from the type annotation on B}
   \mycalloutup<+>{scope}{tracks Matlab scope}
  \end{itemize}
\end{frame}

\begin{frame}[fragile]{Dimensions and Quantities}
  \begin{itemize}[<+->]
  \item \lm has support for arbitrary quantities
    \begin{lstlisting}[xleftmargin=0em,belowskip=-1em]
%> dimensions V for Q over `Mass, `Time, `Length!\tikzmark{base}!
%> unit kg :: Q({!\tikzmark{unit1}! `Mass })!\tikzmark{unit}!
\end{lstlisting}
\pause
   \mycallout<+>{base}{define some base set of dimensions}
   \mycalloutup<+>{unit}{and a canonical unit of measure}
   \mycalloutup<+>{unit1}{can be arbitrary group expression over V}
\pause
\begin{lstlisting}[xleftmargin=0em]
%<{
kg = 1; !\tikzmark{kg}!
%<}
\end{lstlisting}
\pause
   \mycallout<+>{kg}{this is a "magic" response that LabMate emits}
\pause
\item we can then use the unit of measure to create quantities in our program:
\begin{lstlisting}[xleftmargin=0em]
y = 5*kg !\tikzmark{q}!
%> typeof y
%< y :: Quantity (Enum [`Mass, `Time, `Length]) {`Mass}
\end{lstlisting}
  \pause
   \mycallout<+>{q}{turn a value of a dimensionless type into a quantity}
  \end{itemize}
\end{frame}

\begin{frame}[fragile]{Dimensional Consistency for Matrices}
  \begin{itemize}[<+->]
  \item Most \ma program does not work on uniform matrices, the type of the entry $e_{i,j}$ might depend on the indices $i$ and $j$
  \item A common scenario when working with matrices of quantities
%    $\begin{array}{ccccc}
% e_{1,1}  &        & e_{1,j} &        & e_{1,n} \\
%         &        &         &        &         \\
% e_{i,1}  &        & e_{i,j}\tikzmark{mat} &        & e_{i,n} \\
%         &        &         &        &         \\
% e_{m,1}  &        & e_{m,j} &        & e_{m,n}  \\
%\end{array}$
%    \mycallout<+>{mat}{the type might depend on i and j}
\begin{lstlisting}[xleftmargin=0em]
%> dimensions V for Q over `Length, `Mass, `Time
%> unit metre :: Q({ `Length })
%> unit kg :: Q({ `Mass })
%> unit sec :: Q({ `Time })
\end{lstlisting}
\item Work in progress: \lm support for such matrices
\begin{lstlisting}[xleftmargin=0em]
% > x :: [ i <- [{} {`Time}]
%        x j <- [{} {`Length}]
%        ] Q({`Mass * j / i})
x = [ 2*kg       5*kg*metre
      3*kg/sec   4*kg*metre/sec ]
\end{lstlisting}
  \end{itemize}
\end{frame}

\begin{frame}{Implementation Details}
  \begin{itemize}[<+->]
  \item \ma program are modelled as trees of commands, rather than sequence of commands
    \begin{itemize}
    \item the type information is propagated (consistently) throughout the tree; can put type annotations after variable declaration
    \item not every annotation is needed, document the interesting ones
    \end{itemize}
  \item \ma expressions are translated to \lm internal core type theory:
    \begin{itemize}
    \item Matrix types are parametrised over 5 parameters with dependencies between them
    \item Quantities are modelled as the free Abelian group over a base set of dimensions
    \item The typechecker understands some nontrivial algebraic properties
    \end{itemize}
  \end{itemize}
\end{frame}

\begin{frame}{Current Progress \& Future Plans}
  \begin{itemize}[<+->]
  \item \lm is under active development
    \begin{itemize}
    \item available on GitHub, please get it in touch if interested
    \end{itemize}
  \item Work in the pipeline:
    \begin{itemize}
    \item \keyword{Uniqueness of representation}: currently, a matrix with quantities can have more than one corresponding type; this might lead to odd behaviour during typechecking
      \item \keyword{Quality of life improvements}: better messages and more readable responses from \lm
      \end{itemize}
  \item We want to extend our coverage to loops and conditionals in the future.
  \end{itemize}
  \pause
  %\large \centerline{Thank you for your attention!}
\end{frame}



\end{document}
%%% Local Variables:
%%% mode: latex
%%% TeX-master: t
%%% End:
