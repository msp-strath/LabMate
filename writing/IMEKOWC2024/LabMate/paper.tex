\documentclass{ws-procs9x6}

%%%% Packages %%%%%%%%%%%%%%%%%%%%%%%%%%%%%%%%%%%%%%%%%%%%
\usepackage[colorlinks=true]{hyperref}
\usepackage{microtype}
%%%%%%%%%%%%%%%%%%%%%%%%%%%%%%%%%%%%%%%%%%%%%%%%%%%%%%%%%%

%%%% Macros %%%%%%%%%%%%%%%%%%%%%%%%%%%%%%%%%%%%%%%%%%%%%%
\newcommand{\pct}{\texttt{\symbol{37}}}
\newcommand{\dir}{\texttt{\symbol{62}}}
\newcommand{\res}{\texttt{\symbol{60}}}
\newcommand{\lcb}{\texttt{\symbol{123}}}
\newcommand{\rcb}{\texttt{\symbol{125}}}
\newcommand{\lsb}{\texttt{\symbol{91}}}
\newcommand{\rsb}{\texttt{\symbol{93}}}

\newcommand{\remph}{\emph}
%%%%%%%%%%%%%%%%%%%%%%%%%%%%%%%%%%%%%%%%%%%%%%%%%%%%%%%%%%

\begin{document}

\title{LabMate: supporting types for MATLAB}
\author{Conor McBride, Georgi Nakov, Fredrik Nordvall Forsberg, Neil Ghani, Alasdair Forbes, Keith Lines, Ian Smith}
\address{University of Strathclyde, National Physical Laboratory}
%TODO: Affiliations

\bodymatter

\section{Introduction}

% types help

% at low cost  (ie: don't need to start from scratch)

% Matlab popular etc


\section{LabMate in action}

% simple matrix multiplication example

\section{Types for matrices}

% matrix types, and their motivation

% typechecking

\section{Implementation}

% transducer model

% elaboration into core type theory (with richer equational theory than state-of-the-art)

% Stack machine, with disjunctive problem solving

\section{Supporting dimensional consistency}

% matrix multiplication with units of measure

\section{Conclusions}

% future work etc


\bibliographystyle{ws-procs9x6}
%\bibliography{labmate}

\end{document}

%%% Local Variables:
%%% mode: latex
%%% TeX-master: t
%%% End:
